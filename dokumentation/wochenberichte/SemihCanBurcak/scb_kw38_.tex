
\subsubsection{KW38: 20.09.2021 bis 26.09.2021}
\begin{quote}
	\subsubsection*{Arbeit in der Schule}
	Nach der Besprechung mit Professor Rusch, haben wir über das Projekt diskutiert. Wir haben über den Antrag, Pflichtenheft, Verlauf des Projektes (wer was macht, wie man einzelne Meilensteine realisieren wird, welche Schnittstellen werden verwendet, welche Assets werden verwendet, was soll die Geschichte des Spiels sein etc.)
	
	\subsubsection*{Arbeit außerhalb der Schule}
	\begin{itemize}
		\item DA-Antrag, Pflichtenheft - ca 2h
		\item LaTeX und Virtual-Machine einrichten - 1h
		\item Unity Installation und Beginner Tutorials - 2h
	\end{itemize}
	Die Installation von LaTeX machte mir am Anfang Probleme aber ich habe es gelöst, indem ich den ganzen Prozess nochmals durchgemacht habe. DA-Antrag musste ich verbessern und manche Teile neu schreiben, hier habe noch zusätzlich die Sachen zum Spiel geschrieben und die anderen Teile des Projekt wurden von den entsprechenden Mitglieder geschrieben. Am Ende habe ich alles in einer Datei zusammengefasst. Bei dem Lastenheft/Pflichtenheft habe ich mein Teil bezüglich des Spiels und Projektmanagement geschrieben und anschließend es meinen Teamkollegen geschickt. Ich habe das Lastenheft am Sonntag geschickt, was meiner Meinung nach spät ist aber ich werde versuchen mehr diszipliniert zu arbeiten, damit ich meinen Teamkollegen kein Zeitdruck erzeuge.
	
	\subsubsection*{Was ist geplant für die Nächste Woche}
	Diese Woche werde ich mich auf Unity konzentrieren und viele Tutorials durchmachen. Da ich für ein nicht-existierendes Spiel keine Online-Funktion programmieren kann, werde ich schauen, dass wir das Spiel schnell wie möglichst für Online bereitmachen. Ich werde aber natürlich in dieser Zeit auch das Know-How für Onlinespielentwicklung erschaffen.
\end{quote}

