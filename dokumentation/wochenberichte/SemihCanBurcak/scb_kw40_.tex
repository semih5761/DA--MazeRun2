\subsubsection{KW40: 04.10.2021 bis 10.10.2021}
\begin{quote}
	\subsubsection*{Arbeit in der Schule}
	Ich habe in der Schule weiter am Start-Menü geschaffen, ich war auf der Fehlersuche, wieso die Buttons nicht funktionieren. Ich habe dann festgestellt, dass ein EventHandler gefehlt hat, nachdem haben die Buttons funktioniert. Dann wurde die Funktion im Code von Unity nicht erkannt, ich habe die Lösung noch nicht gefunden. Wir haben auch über den Megacard-Kontroller gesprochen, welche Einzelteile wir brauchen, wie man den Akku wieder aufladen kann etc., bei Fragen haben wir uns an den Herrn Professor Zudrell-Koch gewendet. Nach der großen Pause (15:20) haben wir uns mit dem Herrn Professor Rusch in der Klasse getroffen und er hat uns gezeigt, wie man LaTeX und Gitlab verwendet, was sehr hilfreich war.
	
	\subsubsection*{Arbeit außerhalb der Schule}
	Ich habe meine Wochenberichte und Pflichtenheft auf LaTeX übertragen, am Anfang war es ein bisschen Kompliziert aber man kommt schnell im Rhythmus. Ich habe versucht Quellen einzufügen, aber es ist leider nicht gegangen, ich habe dann später entdeckt, dass ich Syntaxfehler hatte aber auch nach der Verbesserung wurde meine eingefügte Quelle im Quellenverzeichnis nicht gezeigt.
	
	Am Montag (11.10.2021) hat mir Herr Professor Rusch gezeigt wie man Quellen einfügt und somit habe ich das Problem lösen können.
	
	Ich habe das Start-Menü in Unity fertig gemacht. Ich musste mit vielen fehlern kämpfen, die Buttons haben auf einmal nicht funktioniert, ich musste ein EventSystem einfügen um das zu lösen. Dann hat Unity mein Code nicht erkannt, die Lösung dafür war: ich musste das GameObject selbst verwenden anstatt der C\#-Datei.
	
	Mehr konnte ich leider nicht machen, weil ich über das Wochenende krank geworden bin. 
	
	\begin{itemize}
		\item Wochenberichte und Pflichtenheft in LaTeX einfügen - 2h
		\item Start-Menü erstellen - 1.5h
	\end{itemize}

	\subsubsection*{Was ist geplant für die Nächste Woche}
	Ich werde schauen, dass ich die Menüs gut designe und über Multiplayerspielentwicklung in Unity (+Webserver) recherchiere.
\end{quote}