

\subsubsection{KW47: 22.11.2021 bis 28.11.2021}
\begin{quote}

	\subsubsection*{Arbeit in der Schule}
		Die Scenen von meinem Partner habe ich in den Ordner, wo meine Scenen sind hineinkopiert, um sie nacher zu verbinden. Abbildung \ref{fig:kw47unityscenes2}.
		Anschließend haben wir folgende Funktionen programmiert: Wenn im Menü auf spielen gedrückt wird, wird der Main-Scene, also das Spiel gestartet. Wenn das Spiel Game-Over wird, wird das Spiel von neu gestartet
	
	\subsubsection*{Arbeit außerhalb der Schule}
	Meine nächste Aufgabe ist es, die Kamera zu programmieren. Mit der Funktion Cinemachine von Unity kann sehr leicht eine dnyamische Kamera programmiert werden. Das Cinemachine Brain verwaltet alle virtuellen Kameras und entscheidet, welcher virtuelle Kamera die eigentliche Kamera folgen soll. In unserem Spiel wird die Kamera unseren Hauptcharakter folgen.
	
	- Das Paket Cinemachine vom PacketManager wurde heruntergeladen und installiert.
	
	- Neue Virutelle Kamera wurde erstellt
	
	- folgende Paramter wurden im Inspector-Window von der virtuellen Kamera geändert:
	
		- Composer: Do Nothing
		
		- Follow: Hauptcharakter
		
		- Transposer: Framing Transposer
		
		- Rotation: x-axis to 45
		
		- Camera distance: 8
		
	- Post Processing Layer erstellt um der Kamera einen Filter zu geben
	
		- Post Process vom Packet Manager heruntergeladen und installiert
		
		- Im Main Camera Inspector Window ein Post Processing Layer erstellt
		
		- Im Post Process Layer component den Mode von No Anti-aliasing zu Fast Approximate Anti-aliasing (FXAA) geändert
		
	- Mit diesem Filter haben wir jetzt eine bessere Qualität.

\end{quote}

